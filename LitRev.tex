\documentclass[journal, a4paper,12pt]{IEEEtran}

\usepackage{graphicx}

%\usepackage{subfigure} 
\usepackage{setspace}

\usepackage{url}        

%\usepackage{stfloats} 

\usepackage{amsmath}    

\usepackage{ragged2e}
\usepackage{graphicx}

\linespread{1.5}

\begin{document}

\begin{titlepage}
	\centering
    \vspace{2cm}
    {\scshape\LARGE\bfseries Physics Studies for a Very High Energy Electron-Proton Collider\par}
	{\scshape\Large\bfseries Literature Review\par}
	\vspace{2cm}
	{\scshape\Large Department of Physics and Astrophysics\par University College London\par}
	\vspace{2cm}
	{\huge\bfseries\par}
	{\large Pranav Rao\par}
	{\large First Supervisor : Prof. Matthew Wing\par Second Supervisor: Dr. Alessio Serafini\par}
	\vspace{10cm}
	Word Count: 

	\vfill
\end{titlepage}

    
\justify

\section{Deep Inelastic Scattering}
%REMEMBER TO ADD A DIAGRAM OF DIS PLEASE
%DO NOT FORGET!!!!

Electron-proton collisions at sufficiently high energies result in the proton breaking up into hadrons, this process is referred to as Deep Inelastic Scattering (DIS). This theory has led to the discovery that protons are composite particles containing 3 valance quarks. While elastic scattering, where the proton is not broken up, do occur at high energies as well, DIS is the dominant collision type at high energies.

The hadronic final state consists of many particles and as a result its invariant mass can have a range of values unlike elastic scattering where the invariant mass of the final state is the mass of the proton. This range of values leads to an additional degree of freedom which requires an additional quantity to describe the kinematics of the event. The quantities used are chosen from a list of Lorentz invariant quantities x, y, $\nu$, and $Q^2$. 

The energies at which DIS occurs are high enough to neglect the mass of the electrons while making predictions, as a result a good approximation of $Q^2$ can be made:
\begin{equation}
	Q^2 = 4E_1E_3sin^2\frac{\theta}{2}
\end{equation}
which is always positive.

Bjorken $x$ is one of the Lorentz invariant quantities that can be used to describe kinematics of DIS. It is given by the following equation: 
\begin{equation}
	x = \frac{Q^2}{Q^2 + W^2 - m_p^2}
\end{equation}
where $Q^2$ is the four-momentum transferred, $W^2$ is the invarient mass of the hadronic state, and $m_p^2$ is the rest mass of the proton squared. $W^2 \geq m_p^2$ because for baryon number to be conserved, the hadronic state must contain one baryon and the proton is the lightest baryon. $W^2 - m_p^2 \geq 0$ along with $Q^2 \geq 0$ gives the range $0 \leq x \leq 1$. Bjorken $x$ is a variable that describes the elasticity of the scattering process, and when $x = 1$ the scattering process is elastic and $Q^2 = W^2$.

The second Lorentz invariant dimentionless quantity used to describe DIS kinetics is the inelasticity $y$ and it is defined as follows:
\begin{equation}
	y = 1-\frac{E_3}{E_1}
\end{equation}
The inelasticity can also be understood as the fractional loss in electron energy which also has a range $0 \leq y \leq 1$.

In situations where it is more convinient to work with the energy difference rather than the fractional energy lost the Lorentz invariant quantity $\nu$ is used, and it is described as
\begin{equation}
	\nu = E_1 - E_3
\end{equation}
and can be understood as the absolute energy lost by the electron in the scattering process.






\end{document}





